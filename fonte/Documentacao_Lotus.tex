\documentclass[12pt,a4paper]{article}
\usepackage[a4paper]{geometry}

\usepackage[T1]{fontenc} % fontes vetoriais
\usepackage[utf8]{inputenc} % codificação do input
\usepackage[brazil]{babel} % português nos títulos automáticos, etc
\usepackage{kpfonts} % fontes com bastante recursos tipográficos
\usepackage{xspace}
\usepackage{epsfig}
\usepackage{ulem}
\usepackage{amssymb}
\usepackage[fixlanguage]{babelbib} % arruma pra português a automatização das referências e tal...
\selectbiblanguage{portuguese}

\def\upo{\textsuperscript{\d o}} % definição de macro:
\def\upa{\textsuperscript{\d a}\xspace} % macro com espaço dinâmico (interpreta se deve por ou não, como não antes de pontos finais).

\def\emph#1{\textbf{#1}} % #1 é o argumento

\begin{document}
\begin{titlepage}
\begin{center}
{\large Universidade Federal da Fronteira Sul}\\[5.5cm]
{\bf \huge Linguagem de programação Lotus}\\[4.9cm]
\end{center}
{\large Acácia dos Campos da Terra}\\
{\large Gabriel Batista Galli}\\
{\large Vladimir Belinski}\\[5.8cm]
\begin{center}
{\large Chapecó}\\[0.1cm]
{\large 2015}
\end{center}
\end{titlepage}
\end{document}
