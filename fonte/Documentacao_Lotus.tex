\documentclass[12pt,a4paper]{article}
\usepackage[a4paper]{geometry}

\usepackage[T1]{fontenc} % fontes vetoriais
\usepackage[utf8]{inputenc} % codificação do input
\usepackage[brazil]{babel} % português nos títulos automáticos, etc
\usepackage{kpfonts} % fontes com bastante recursos tipográficos
\usepackage{xspace}
\usepackage{epsfig}
\usepackage{ulem}
\usepackage[pdftex]{hyperref}
\usepackage{amssymb}
\usepackage[fixlanguage]{babelbib} % arruma pra português a automatização das referências e tal...
\selectbiblanguage{portuguese}

\def\upo{\textsuperscript{\d o}} % definição de macro:
\def\upa{\textsuperscript{\d a}\xspace} % macro com espaço dinâmico (interpreta se deve por ou não, como não antes de pontos finais).

\def\emph#1{\textbf{#1}} % #1 é o argumento

\begin{document}
\begin{titlepage}
\begin{center}
{\large Universidade Federal da Fronteira Sul}\\[5.5cm]
{\bf \huge Linguagem de programação Lotus}\\[4.9cm]
\end{center}
{\large Acácia dos Campos da Terra}\\
{\large Gabriel Batista Galli}\\
{\large Vladimir Belinski}\\[5.8cm]
\begin{center}
{\large Chapecó}\\[0.1cm]
{\large 2015}
\end{center}
\end{titlepage}

A linguagem de programação Lotus é uma linguagem desenvolvida com base em Java com as seguintes características e funcionalidades:\\

\begin{itemize}
\item \hyperlink{Declaração de variáveis}{Declaração de variáveis}
\item \hyperlink{Entrada e saída}{Entrada e saída}
\item \hyperlink{Atribuições de valores}{Atribuições de valores}
\item \hyperlink{Operações aritméticas}{Operações aritméticas}
\item \hyperlink{Desvios condicionais}{Desvios condicionais}
\item \hyperlink{Laços de repetição}{Laços de repetição}
\item \hyperlink{Sintaxe}{Sintaxe}
\item \hyperlink{Extensão de arquivos}{Extensão de arquivos}\\[15.4cm]
\end{itemize}

\hypertarget{Declaração de variáveis}{\Large{Declaração de variáveis}}\\[0.3cm]
\normalsize

A sintaxe de declaração de variável funciona da seguinte forma:\\

let nome\_variavel: tipo\_variavel;\\

Sendo obrigatório o uso de ``let ''\ antes de qualquer declaração e o uso de `;'\ ao final dela. É permitida a criação de várias variáveis do mesmo tipo em apenas uma linha, desde que separadas por vírgulas, exemplo:\\

let x, y, z: int;\\

Também é permitida a atribuição de valores às variáveis no momento da declaração. Exemplo:\\

let x = ``Oi'': string;\\


Expressões:

Todos os tipos operam entre si, através da conversão de tipos, utilizando os seguintes operadores:

% Matemáticos: -, +, /, \%, *, ^ (potência), v (ainda não implementado, possível raiz quadrada);
% Comparação: <, <=, ==, >=, >, !=;
% Booleanos: !, &&, ||.

Os operadores de comparação e os booleanos sempre resultam em um bool. A comparação de strings é em relação a ordem lexicográfica. Se as strings têm tamanhos diferentes, a menor vem antes. Comparações com bool são feitas de modo que o valor é primeiro transformado em int e então é feita a comparação (exceto para os operadores ``=='' e ``!='').

Os operadores matemáticos variam com seu tipo:
Int e double funcionam normalmente, como manda a matemática;
Para booleanos, os operadores matemáticos são interpretados como as seguintes operações lógicas:\\[0.2cm]
$+$ $\rightarrow$ or\\
$-$ $\rightarrow$ nor\\
$/$ $\rightarrow$ nand\\
\textbackslash\ $\rightarrow$ xnor\\
$*$ $\rightarrow$ and\\
\^ \ $\rightarrow$ xor\\

Para string, em relação aos operandos:
%+ concatena o segundo ao primeiro;
%- retira a primeira ocorrência do segundo no primeiro;
%/ retira todas as ocorrências do segundo no primeiro;
%\% retira todas as ocorrências do primeiro no segundo (``resto'' da operação de `/');
%* adiciona o segundo entre cada caractere do primeiro;
%^ ainda não está definido. Lança uma exceção.\\

\hypertarget{Entrada e saída}{\Large{Entrada e saída}}\\[0.3cm]
\normalsize

A função print() é utilizada para imprimir uma linha, sem quebra no final. É equivalente ao printf() do C ou System.out.print() do java. Já o comando println() pode ser usado para quebrar automaticamente uma linha ao final da string a ser impressa (equivalente ao System.out.println() do Java). Os caracteres especiais são:\\[0.2cm]
\textbackslash t para uma tabulação \\
\textbackslash n para uma quebra de linha \\
\textbackslash \$ para um cifrão \\
\textbackslash \textbackslash para uma contra-barra (para imprimir o literal ``\textbackslash n"", faça ``\textbackslash n'') \\


De maneira análoga às funções de impressão, o comando scan() é utilizado para ler um valor do teclado. Esse comando lê a informação até o primeiro espaço ou quebra de linha que delimitarem a entrada desejada, assim como o scanf() do C. Já o comando scanln() lê uma linha inteira, como o fgets() do C. Em ambos os comandos, a conversão dos tipos de variável se aplica, caso a entrada não corresponda ao valor esperado.

Nas funções de impressão, agora é possível imprimir expressões diretamente, desde que ela esteja entre cifrões (`\$'), assim como uma variável. Para escapar uma variável ou expressão, deve-se escapar o primeiro ou ambos os cifrões. Ao escapar apenas o último, o que está entre os cifrões será interpretado como uma expressão, ocasionando em um erro de expressão inválida, já que o operador `\' não está definido.\\

\hypertarget{Atribuições de valores}{\Large{Atribuições de valores}}\\[0.3cm]
\normalsize

\hypertarget{Operações aritméticas}{\Large{Operações aritméticas}}\\[0.3cm]
\normalsize

\hypertarget{Desvios condicionais}{\Large{Desvios condicionais}}\\[0.3cm]
\normalsize

A linguagem suporta toda e qualquer cadeia ou aninhamento de comandos if. A condição é qualquer expressão suportada pela linguagem, cujo resultado será transformado em booleano para efeitos de avaliação.\\

Um comando de desvio condicional se comportará da seguinte forma: \\

if (\emph{condição})\{ \\

\emph{comandos a serem executados quando a condição for verdadeira} \\

\} elseif (\emph{outra condição})\{\\

\emph{comandos a serem executados quando a segunda condição for verdadeira} \\

\} else \{\\


\emph{comandos a serem executados quando nenhuma das condições acima for verdadeira} \\

\}\\


Sendo "if", "else"\ e "elseif"\ palavras reservadas para as operações de desvio e o uso de `(' e `)' obrigatórios para delimitar a(s) condição(ões). O uso de `$\{$' e `$\}$' também é obrigatório para delimitar o bloco de operação.\\

% O comando "if" pode conter uma ou mais expressões para análise

\hypertarget{Laços de repetição}{\Large{Laços de repetição}}\\[0.3cm]
\normalsize

% A fazer

\hypertarget{Sintaxe}{\Large{Sintaxe}}\\[0.3cm]
\normalsize

Os comandos podem apresentar quebras de linha que podem incluir até mesmo comentários, caso seja conveniente. Todas as linhas com comandos devem ser finalizadas com `;'.

Já comandos do tipo if, for e while exigirão `{' e `}' como delimitadores de bloco.
Comentários são marcados com ``--'', assim como o ``//'' do C ou Java. Não há comentários de bloco.\\


\hypertarget{Extensão de arquivos}{\Large{Extensão de arquivos}}\\[0.3cm]
\normalsize

Os códigos escritos com a linguagem de programação Lotus devem apresentar extensão ``.lt''.\\


Conversão de tipos:

A partir de int:
- pra double, ocorre normalmente: se meu int era x, o double será x.0;
- pra bool: se o int era 0, torna-se false, senão true;
- pra string: se torna uma string que representa o inteiro.

A partir de double:
- pra int: resumidamente, o valor é truncado;
- pra bool: se o valor era 0.0, torna-se false, senão true;
- pra string: assim como de int, torna-se uma string que representa o double.

A partir de bool:
- pra int: se era true, torna-se 1, senão 0;
- pra double: se era true, torna-se 1.0, senão 0.0;
- pra string: torna-se uma string de conteúdo ``true'' ou ``false''.

A partir de string:
- pra int e double:
Se a string representa um int ou double, converte-se normalmente. Exemplo: ``7'' se torna 7 e ``11.0'' se torna 11.0;
Se a string é exatamente ``true'' ou ``false'', a conversão de bool pra int/double ocorrerá;
Caso contrário, o int ou double receberão o tamanho da string, para evitar exceptions.
- pra bool: será true se a string é exatamente ``true''. Caso contrário, false.

\end{document}
