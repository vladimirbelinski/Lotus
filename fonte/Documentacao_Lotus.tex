\documentclass[12pt,a4paper]{article}
\usepackage[a4paper]{geometry}

\usepackage[T1]{fontenc} % fontes vetoriais
\usepackage[utf8]{inputenc} % codificação do input
\usepackage[brazil]{babel} % português nos títulos automáticos, etc
\usepackage{kpfonts} % fontes com bastante recursos tipográficos
\usepackage{xspace}
\usepackage{xcolor}
\usepackage{epsfig}
\usepackage{ulem}
\usepackage{hyperref}
\usepackage{amssymb}
\usepackage[fixlanguage]{babelbib} % arruma pra português a automatização das referências e tal...

\hypersetup{colorlinks=false,linkbordercolor=red,linkcolor=green,pdfborderstyle={/S/U/W 1}}
\selectbiblanguage{portuguese}

\def\upo{\textsuperscript{\d o}} % definição de macro:
\def\upa{\textsuperscript{\d a}\xspace} % macro com espaço dinâmico (interpreta se deve por ou não, como não antes de pontos finais).

\def\emph#1{\textbf{#1}} % #1 é o argumento

\begin{document}
\begin{titlepage}
\begin{center}
{\large Universidade Federal da Fronteira Sul- UFFS}\\[5.5cm]
{\bf \huge Linguagem de programação Lotus}\\[4.9cm]
\end{center}
{\large Acácia dos Campos da Terra}\\
{\large Gabriel Batista Galli}\\
{\large Vladimir Belinski}\\[5.8cm]
\begin{center}
{\large Chapecó}\\[0.1cm]
{\large 2015}
\end{center}
\end{titlepage}

A linguagem de programação Lotus é uma linguagem desenvolvida com base em Java com as seguintes características e funcionalidades:\\

\begin{itemize}
\item \hyperlink{label7}{Extensão de arquivos}
\item \hyperlink{label}{Declaração de variáveis}
\item \hyperlink{label1}{Entrada e saída}
\item \hyperlink{label2}{Atribuições de valor}
\item \hyperlink{label3}{Operações aritméticas}
\item \hyperlink{label4}{Desvios condicionais}
\item \hyperlink{label5}{Laços de repetição}
\item \hyperlink{label6}{Sintaxe Flexível}
\item \hyperlink{label8}{Conversão de tipos}\\[15.4cm]
\end{itemize}

\hypertarget{label7}{\Large{Extensão de arquivos}}\\[0.3cm]
\normalsize

Os códigos escritos com a linguagem de programação Lotus devem apresentar extensão ``.lt''.\\

\hypertarget{label}{\Large{Declaração de variáveis}}\\[0.3cm]
\normalsize

A sintaxe de declaração de variável funciona da seguinte forma:\\

let nome\_variavel: tipo\_variavel;\\

Sendo obrigatório o uso de ``let '' antes de qualquer declaração e o uso de `;'\ ao final dela. É permitida a criação de várias variáveis do mesmo tipo em apenas uma linha, desde que separadas por vírgulas, exemplo:\\

let x, y, z: int;\\

Também é permitida a atribuição de valores às variáveis no momento da declaração. Exemplo:\\

let x = ``Oi'': string;\\

\hypertarget{label1}{\Large{Entrada e saída}}\\[0.3cm]
\normalsize

Para imprimir uma linha, deve ser chamada a função print(). A mesma é equivalente ao printf() do C ou System.out.print() do java.\\
O comando println() imprimirá o que foi solicitado e também imprimirá uma quebra de linha ao final (equivalente ao System.out.println() do Java).\\

Para imprimir o valor de uma variável, a mesma deve ser expressa entre cifrões. Exemplo: \\

let ex: string;\\

print(var: \$ex\$); \\

Onde ``print'' é o comando e o uso de `$($' e `$)$' é obrigatório para delimitar o comando. ``var: '' é uma string a ser posta diretamente na tela e \$ex\$ é o nome da variável, que está delimitada pelos cifrões para poder ser impressa.\\

Os caracteres especiais em um comando de saída são:\\[0.2cm]
\textbackslash t para uma tabulação \\
\textbackslash n para uma quebra de linha (para imprimir o literal ``\textbackslash n'', faça ``\textbackslash \textbackslash n'')\\
\textbackslash \$ para um cifrão \\
\textbackslash \textbackslash\ para uma contra-barra \\

Os comandos ``print'' e ``println'' aceitam também expressões em sua forma pura, desde que delimitadas pelos cifrões, como se fosse uma variável e então as expressões são resolvidas. Exemplo: \\

println(\$$\ -3\ +4\ *2\ /\ (\ 1\ - 5)$\^\ 2 \^\ 3\$);\\

O qual a resposta será $-3$.\\

De maneira análoga às funções de saída, o comando scan() é utilizado para ler um valor do teclado. Esse comando lê a informação até o primeiro espaço ou quebra de linha que delimitar a entrada desejada, sendo idêntico ao scanf() do C. Já o comando scanln() lê uma linha inteira, assim como o fgets() do C. Em ambos os comandos, a conversão dos tipos de variável se aplica, caso a entrada não corresponda ao valor esperado.\\

Exemplo do comando scan(): \\

let i: int;\\

scan(i);\\

Onde ``scan'' é o comando, `$($' e `$)$' são os caracteres obrigatórios de delimitação e ``i'' é a variável onde será armazenado o valor lido. Neste exemplo, um valor do tipo inteiro.\\

% Para escapar uma variável ou expressão, deve-se escapar o primeiro ou ambos os cifrões. Ao escapar apenas o último, o que está entre os cifrões será interpretado como uma expressão, ocasionando em um erro de expressão inválida, já que o operador `\' não está definido.\\

\hypertarget{label2}{\Large{Atribuições de valor}}\\[0.3cm]
\normalsize

Para atribuir valores à uma variável, basta fazer: \\

nome\_variavel = valor\_variavel; \\

As atribuições também podem ser feitas na própria declaração de variável, como demonstrado. \\

Se por ventura for atribuído um valor diferente do do tipo da variável, então será automaticamente convertido. Exemplo:\\

let b: bool; \\

b = 861;\\

b após a conversão automática valerá ``true''.\\

\hypertarget{label3}{\Large{Operações aritméticas}}\\[0.3cm]
\normalsize

Todos os tipos de variáveis operam entre si, através da conversão de tipos, utilizando os seguintes operadores:\\[0.2cm]
Matemáticos: $-,\ +,\ /,\ \%,\ *,$ \^\\
Comparação: $<,\ <=,\ ==,\ >=,\ >,\ !=$\\
Booleanos: $!,\ \&\&,\ ||$\\[0.6cm]
Exemplo de uso:\\[0.3cm]
let x: int;\\
let y: double;\\

x = $3--4$;\\

y = $+\ 2$ \^\ $\ -\ 4\ +\ -0\ /\ +\ 5$;\\[0.5cm]
Onde a resposta será x = 7 e y = 0.0625.\\


Os operadores de comparação e os booleanos sempre resultam em um booleano. A comparação de strings é em relação a ordem lexicográfica (se as strings têm tamanhos diferentes, a menor vem antes). Comparações com booleanos são feitas de modo que o valor é primeiro transformado em inteiro e só então é feita a comparação (exceto para os operadores ``=='' e ``!='').\\

Os operadores matemáticos variam com seu tipo:\\
Int e double funcionam normalmente, como ensinado na matemática.\\

Para booleanos, os operadores matemáticos são interpretados como as seguintes operações lógicas:\\[0.2cm]
$+$ $\rightarrow$ or\\
$-$ $\rightarrow$ nor\\
$/$ $\rightarrow$ nand\\
\% $\rightarrow$ xnor\\
$*$ $\rightarrow$ and\\
\^ \ $\rightarrow$ xor\\

Para string, em relação aos operandos:\\[0.2cm]
$+$ $\rightarrow$ concatena o segundo ao primeiro\\
$-$ $\rightarrow$ retira a primeira ocorrência do segundo no primeiro\\
$/$ $\rightarrow$ retira todas as ocorrências do segundo no primeiro\\
\% $\rightarrow$ retira todas as ocorrências do primeiro no segundo (``resto'' da operação de `/')\\
$*$ $\rightarrow$ adiciona o segundo entre cada caractere do primeiro\\
\^ \ $\rightarrow$ se o segundo operando for uma string, retornará o índice daquela string dentro do primeiro operando. Caso contrário, retorna o char do primeiro na posição indicada pelo segundo\\

\hypertarget{label4}{\Large{Desvios condicionais}}\\[0.3cm]
\normalsize

A linguagem suporta toda e qualquer cadeia ou aninhamento de comandos if. A condição é qualquer expressão suportada pela linguagem, cujo resultado será transformado em booleano para efeitos de avaliação.\\

Um comando de desvio condicional se comportará da seguinte forma: \\

\noindent\texttt{if (\emph{condição}) \{\\
\indent\textit{-- comandos a serem executados quando a condição for verdadeira} \\
\} elsif (\emph{outra condição}) \{\\
\indent\textit{-- comandos a serem executados quando a segunda condição for verdadeira} \\
\} else \{\\
\indent\textit{-- comandos a serem executados quando nenhuma das condições acima for verdadeira} \\
\}}\\

Sendo ``if'', ``elsif'' e ``else'' palavras reservadas para as operações de desvio e o uso de `(' e `)' obrigatórios para delimitar a(s) condição(ões). O uso de `$\{$' e `$\}$' também é obrigatório para delimitar o bloco de código, mesmo que seja escrito em uma linha.\\

% O comando ``if'' pode conter uma ou mais expressões para análise

\hypertarget{label5}{\Large{Laços de repetição}}\\[0.3cm]
\normalsize

% A fazer

\hypertarget{label6}{\Large{Sintaxe Flexível}}\\[0.3cm]
\normalsize

Os comandos podem apresentar quebras de linha que podem incluir até mesmo comentários, caso seja conveniente. Todas as linhas com comandos devem ser finalizadas com `;'.

Já comandos do tipo if, for e while exigirão `\{' e `\}' como delimitadores de bloco.
Comentários são marcados com ``-{}-'', assim como o ``//'' do C ou Java. Não há comentários de bloco.\\

\hypertarget{label8}{\Large{Conversão de tipos}}\\[0.3cm]
\normalsize

Todas as variáveis que receberem um valor que não seja compatível com o seu próprio tipo (especificado na declaração) passarão por uma conversão automática, onde: \\

Se foi atribuído um número inteiro:\\
- à uma variável double: se o int era x, o double será x.0\\
- à uma variável bool: se o int era 0, torna-se false, senão true\\
- à uma variável string: se torna uma string que representa o valor inteiro\\[0.3cm]

Se for atribuído um número double:\\
- à uma variável int: o valor será truncado, ou seja, só será guardado o que está antes da vírgula\\
- à uma variável bool: se o valor for 0.0, torna-se false, senão true\\
- à uma variável string: torna-se uma string que representa o valor double\\[0.3cm]

Se for atribuído um valor bool:\\
- à uma variável int: se for true, torna-se 1 e se false, torna-se 0\\
- à uma variável double: se for true, torna-se 1.0, se false, 0.0\\
- à uma variável string: torna-se uma string de conteúdo ``true'' ou ``false'', dependendo do valor da variável\\[0.3cm]

Se for atribuído um valor de uma string:\\
- à uma variável int ou double: Se a string representa um int ou double, converte-se normalmente. Exemplo: ``7'' se torna 7 e ``11.0'' se torna 11.0\\
Se a string for exatamente ``true'' ou ``false'', a conversão de bool pra int/double ocorrerá\\
Caso contrário, o int ou double receberão o tamanho da string, para evitar exceções no programa.\\
- à uma variável bool: será true se a string é exatamente ``true''. Caso contrário, false.\\

\end{document}
